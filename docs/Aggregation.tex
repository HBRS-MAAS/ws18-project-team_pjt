\section{Aggregation of order data}
Aggregation of order data can be done in the following manner:
\begin{itemize}
    \item \textbf{An aggregation of a customer’s orders for each day or each date <ddd.hh>} $\rightarrow$ It depends which data structure you could use
    \begin{itemize}
        \item If it is really important to you that you access date by given data format you could use a hashmap. Key is date value is order.
        Worst performance of searching a hashmap is $O(n) = log(n)$
        {\footnotesize \begin{lstlisting}
            Hashmap<Date, Order> hmMapDaily = new Hashmap<Date, Order>();
            hmMapDaily.put(new Date(), new Order());
            Order co = hmMapDaily.get(date);
        \end{lstlisting}}
        \item If it is not that important to use the given dateformat you could use an array. Index is day of a year. That means here worst performance of searching an array
        given that you know which day you want to search is $O(n) = 1$
    \end{itemize}
    \item \textbf{An aggregation of all orders for a particular product for each day or each date} \\ $\rightarrow$
    Hashmap of Hashmaps. One entry within Hashmap represents one product. Key is product value is a hashmap. One Hashmap within Hashmap has
    as key a date, as value an array of orders.
{\footnotesize \begin{lstlisting}
Hashmap<ProductId, Hashmap<Date, Orders[]>> hMapProduct;
hMapProduct.put(new ProductId(), Hashmap<Date, Orders[]>);
Hashmap<Date, Orders[]> hmDate = hMapProduct.get(ProductId);
\end{lstlisting}}
    So hMapProduct would look the following way:
    \[hMapProduct = \begin{pmatrix}
    \{ProductId, Hashmap<Date, Orders[]>\} \\
         . \\
         . \\
         . \\
    \{ProductId, Hashmap<Date, Orders[]>\}
    \end{pmatrix}\]
\end{itemize}

