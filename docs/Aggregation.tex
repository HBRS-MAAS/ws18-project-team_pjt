\chapter*{Aggregation of order data}
Aggregation of order data can be done in the following manner:
\begin{itemize}
    \item \textbf{An aggregation of a customer’s orders for each day or each date <ddd.hh>} \\ $\rightarrow$
    Use of a Hashmap. Key is date value is order. The advantage is that a hashmap has got an index.
    That means that worst case runtime for searching for an order within hashmap is $O(n) = 1$
    \begin{lstlisting}
Hashmap<Date, Order> hmMapDaily = new Hashmap<Date, Order>();
hmMapDaily.put(new Date(), new Order());
Order co = hmMapDaily.get(date);
    \end{lstlisting}
    \item \textbf{An aggregation of all orders for a particular product for each day or each date} \\ $\rightarrow$
    Hashmap of Hashmaps. One entry within Hashmap represents one product. Key is product value is a hashmap. One Hashmap within Hashmap has
    as key a date, as value an array of orders.
    \begin{lstlisting}
Hashmap<ProductId, Hashmap<Date, Orders[]>> hmMapProduct;
hmMapProduct.put(new ProductId(), Hashmap<Date, Orders[]>);
Hashmap<Date, Orders[]> hmDate = hmMapProduct.get(ProductId);
    \end{lstlisting}
    So hmMapProduct would look the following way:
    \[hmMapProduct = \begin{pmatrix}
    \{ProductId, Hashmap<Date, Orders[]>\} \\
         . \\
         . \\
         . \\
    \{ProductId, Hashmap<Date, Orders[]>\}
    \end{pmatrix}\]
\end{itemize}