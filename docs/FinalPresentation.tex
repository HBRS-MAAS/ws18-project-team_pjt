\documentclass[hyperref={pdfpagelabels=false}]{beamer}
%\usepackage{href}
\usepackage{listings}
\usepackage[utf8]{inputenc}
\setbeamertemplate{section in toc}[sections numbered]
\setbeamertemplate{subsection in toc}[subsections numbered]
\newcommand{\tit}{MAAS Project}
\newcommand{\ti}{Technische Anforderungen}
\newcommand{\tzi}{Techniken zur Implementierung}
\newcommand{\wui}{WebUserInterface}
\newcommand{\fa}{Funktionale Anforderungen}
\newcommand{\arch}{Architekturüberblick}
\newcommand{\cl}{Client}
\newcommand{\ml}{Middleware}
\newcommand{\al}{Application Server}
\newcommand{\re}{Service Registry}
\newcommand{\db}{Datenbank}
%\usepackage[utf8]{inputenc}
\usepackage{geometry}
%\geometry{bottom=0.9in}
\usepackage{fancyhdr}
\setbeamertemplate{footline}[text line]{
\parbox{\linewidth}{\vspace*{-20pt} \hyperlink{tableofcontent}{\includegraphics[scale=0.03]{./images/rhein-sieg.jpg}} \hspace{1cm} \tit \hfill\insertshortauthor\hfill\insertpagenumber}}
\setbeamertemplate{navigation symbols}{}
\author{Team PJT}
\title{\tit}
\usepackage{lmodern}
\setcounter{tocdepth}{1}


\begin{document}
    %\lstset{language=Python}
    %\lstset{language=Python}

    \begin{frame}
        \titlepage
    \end{frame}

    \begin{frame}
        \frametitle{Agenda}
        \hypertarget{tableofcontent}{}
        \tableofcontents
    \end{frame}

    \section{Dough Manager}
    %\label{anforderung}
    \begin{frame}
        \frametitle{Dough Manager}
        \begin{itemize}
            \item The Dough Manager is responsible for kneading and preparing the product
            \item It manages its assigned prep tables and kneading machines which means
            \begin{itemize}
                \item moving one order from one state to next state
                \item check whether current stage of dough preparing is finisehd
                \item receiving orders from order processing agent
                \item sending orders to proofer
            \end{itemize}
        \end{itemize}
    \end{frame}


    %\label{anforderung}
    \section{Kneading Machine and Preparation Table}
    \begin{frame}
        \frametitle{Kneading Machine and Preparation Table}
        \begin{itemize}
            \item They are represented as an object based on the same class
            \item The way we designed them there is not much functionality inside preparing machine and kneading machine
            \item Their only task it to update a counter and check whether a certain time threshold is reached
%            \begin{itemize}
%                \item moving one order from one state to next state
%                \item check whether current stage of dough preparing is finisehd
%                \item receiving orders from order processing agent
%                \item sending orders to proofer
%            \end{itemize}
        \end{itemize}
    \end{frame}

    \section{Proofer}
    \begin{frame}
        \frametitle{Proofer}
        \begin{itemize}
            \item Represents the interface between baking stage and dough preparation stage
            \item Receives kneaded and prepared products from Dough Manager
%            \item Their only task it to update a counter and check whether a certain time threshold is reached
            %            \begin{itemize}
            %                \item moving one order from one state to next state
            %                \item check whether current stage of dough preparing is finisehd
            %                \item receiving orders from order processing agent
            %                \item sending orders to proofer
            %            \end{itemize}
        \end{itemize}
    \end{frame}


%    \section{Technische Anforderung}
%    \begin{frame}
%        \frametitle{Technische Anforderungen}
%        Für folgenden technischen Ausschnitt der Anforderungen wurde sich entschieden
%        \begin{itemize}
%            \item State-of-The Art Ansatz zur Entwicklung der notwendigen Client-					Anwendung
%            \item Lösungsstrategie für die Cloud
%            \item Optimierung der Skalierbarkeit und der Ausfallsicherheit
%        \end{itemize}
%    \end{frame}
%
%    \section{Architekturüberblick}
%    %\label{architekturueberblick}
%    \begin{frame}
%        \frametitle{\arch}
%        \begin{center}
%            \href{file:///Users/janloeffelsender/Library/Mobile Documents/com~apple~CloudDocs/Studium/18_Sommersemester/SOA/Semesterprojekt/scripts/displayallcontainer.app}	{\includegraphics[scale=0.18]{../4-Sichten-Modell/Verteilungssicht.jpg}}
%        \end{center}
%    \end{frame}
%    \section{Orchestrierung Docker Container}
%    \begin{frame}
%        \begin{center}
%            \includegraphics[scale=0.17]{graphics/yamlfile.jpg}
%        \end{center}
%    \end{frame}
%    \section{Client}
%    %\label{client}
%    \begin{frame}
%        \frametitle{\cl}
%        \begin{center}
%            \href{http://localhost:4200/}{\includegraphics[scale=0.25]{graphics/client.jpg}}
%        \end{center}
%    \end{frame}
%    \section{Application Server}
%    %\label{middleware}
%    \begin{frame}
%        \frametitle{\al}
%        \begin{center}
%            \href{http://localhost:8080/service-instances/A-BOOTIFUL-CLIENT/findallbill}	{\includegraphics[scale=0.4]{graphics/ApplikationServer.jpg}}
%        \end{center}
%    \end{frame}
%    \section{Service Registry}
%    %\label{serviceregistry}
%    \begin{frame}
%        \frametitle{\re}
%        \begin{center}
%            \href{http://localhost:8761/}{\includegraphics[scale=0.4]{graphics/serviceregistry.jpg}}
%        \end{center}
%    \end{frame}
%    %\subsection{Middleware}
%    \section{Datenbank}
%    %\label{datenbank}
%    \begin{frame}
%        \frametitle{\db}
%        \begin{center}
%            \href{file:///Users/janloeffelsender/Library/Mobile Documents/com~apple~CloudDocs/Studium/18_Sommersemester/SOA/Semesterprojekt/scripts/displaymongodb.app}{\includegraphics[scale=0.4]{graphics/mongoDB.jpg}}
%        \end{center}
%    \end{frame}
%    \section{Ausfallsicherheit}
%    \begin{frame}
%        \frametitle{Ausfallsicherheit}
%        Um die Ausfallsicherheit zu demonstrieren, wird der Lese-Service Container heruntergefahren
%        \begin{itemize}
%            \item [1.] \href{http://localhost:4200/}{\color{blue}Öffnen} des WebClient - 			Änderungen eintragen
%            \item [2.] \href{file:///Users/janloeffelsender/Library/Mobile Documents/com~apple~CloudDocs/Studium/18_Sommersemester/SOA/Semesterprojekt/scripts/stopReadContainer.app}{\color{blue}Herunterfahren} vom Docker-Container, der den LeseService \href{http://localhost:8761/}{\color{blue}verwaltet}
%            \item [3.] Festschreiben der Änderungen
%        \end{itemize}
%    \end{frame}

    %\section{\ti}
    %	\begin{frame}
    %		\frametitle{\ti}
    %		\begin{itemize}
    %			\item Verwendete technische Anforderung:
    %		\end{itemize}
    %	\end{frame}
    %
    %\section{\tit}
    %\begin{frame}
    %		\frametitle{\tit}
    %		\begin{center}
    %			Ein Neuronales Netzwerk ist ein parallel verteilter Prozessor, bestehend aus einfachen Verarbeitungseinheiten, die die Fähigkeit haben Wissen zu 						speichern.
    %		\end{center}
    %	\end{frame}
    %
    %\section{\wui}
    %\begin{frame}
    %		\frametitle{\wui}
    %		\begin{center}
    %			Ein Neuronales Netzwerk ist ein parallel verteilter Prozessor, bestehend aus einfachen Verarbeitungseinheiten, die die Fähigkeit haben Wissen zu 						speichern.
    %		\end{center}
    %	\end{frame}

\end{document}
